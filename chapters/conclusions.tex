Functional brain connectivity, measured by fMRI, greatly increased the knowledge of the architecture and brain.
In particular, investigation of the modular organization of brain functional connectivity networks, is specifically explored in this thesis by means of a new graph theoretical approach.
Cohesive clusters of tightly interacting brain regions can be found within the widespread functional connectivity network, representing functionally segregated modules of brain regions which are strongly related to each other.

In the past, these functional networks have been investigated by means of methods that were hampered by a limit that impeded the detection of modules smaller than a scale determined, in the best case, by the amount of interconnections and, in the worst case, by the overall size of the graph under consideration.

In this work I pioneered an approach for community detection in brain networks that is free from these sever limitations. Validation and testing of the Surprise approach to functional networks has been examined both from the theoretical and computational point of view, as well as with real world networks.

The methodological advantage of the approach proposed in this thesis is two-fold.
Firstly, the resolution limit free method for community detection offered a richer view on the modular structure of brain networks.
A 
Direct comparison with other methods clearly demonstrated improved capability to identify neurofunctionally plausible and anatomically well-defined substructures otherwise concealed by the resolution limit.

Functional connectivity alterations have been observed in several neuropsychiatric and neurodegenerative disorders including schizophrenia, autism and Alzheimer's disease. The dysconnectivity modifications suffered from the diseased brain are thought to reflect on the modular hierarchical structure of its functional connections.
Therefore, the greater resolution afforded by these new tools can enhance researchers swiss-knife, helping them to characterize the properties of modules not only when posed in relation to each other, but especially when compared in relation diseases.


 

Unfortunately Newman's Modularity, the most widely used graph-based method for community detection in FC networks is flawed as it cannot detect modules that are smaller than a certain scale, making impossible to ascertain important details of architectural structure.
The two main drawbacks of Modularity have been introduced namely the resolution limit and its extreme degeneracy, together with analyses that specifically address their origin.
In the following chapter I will illustrate how to overcome the limitations imposed by Newman's Modularity with an approach that has its roots in probability theory: Surprise.


Complex networks theory offers a framework for the analysis of functional connectivity.
Within this approach the brain is represented as a graph comprising N nodes connected by M edges.
The nodes typically correspond to brain regions and the edges to measures of inter-regional interaction.
Functionally connected networks encompass the entire brain as a consequence of the high level of integration of brain function.

A number of graph theoretical methods have been proposed to analyze the modular structure of these networks.
The most widely used metric is Newman's Modularity, which identifies modules within which links are more abundant than expected on the basis of a random network.
However, Modularity is limited in its ability to detect relatively small communities, a problem known as “resolution limit”.
As a consequence, unambiguously identifiable modules, like complete sub-graphs, may be unduly merged into larger communities when they are too small compared to the size of the network.
Hence, the resolution limit may represent a critical shortcoming for the study of brain networks, and is likely to have affected many of the studies reported in the literature.
In this work we explore the use of Surprise, a fitness measure grounded in probability theory, that aims at overcoming these limitations, allowing to detect anatomically and functionally plausible modules from brain connectivity datasets beyond the resolution limit.
We then identify a series of properties of the Surprise function that enable efficient implementation of a maximization algorithm, thus making partitioning of relatively large networks, like those routinely built from neuroimaging data, computationally tractable.
Finally, we exploit this novel approach to partition two different human brain networks that have been extensively characterized in the literature, to conclusively address the resolution-limit issue in the study of the brain modular structure.