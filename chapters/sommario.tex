Fin dal loro inizio, le neuroscienze hanno investigato il cervello a diverse scale, partendo dalla neurobiologia molecolare fino ad arrivare alle scienze cognitive.
Nel corso degli anni si è sempre più rafforzata l'idea che per ottenere una comprensione profonda del cervello, sia necessario un approccio multiscala, dove l'analisi del singolo elemento costitutivo sia accompagnata da una conoscenza delle sue interazioni all'interno del sistema.

Fra i metodi di neuroimaging più interessanti, la risonanza magnetica funzionale (fMRI) si è dimostrata foriera di scoperte innovative che hanno fatto da volano allo sviluppo di algoritmi di indagine sofisticati e capaci di fornire nuovi e profondi sguardi nella complessità del cervello.
In particolare, diversi approcci quantitativi a vari livelli sono stati applicati al campo di indagine della neuroradiologia.

Attraverso misure della cosiddetta connettività funzionale è possibile quantificare la correlazione temporale di eventi neurofisiologici in aree cerebrali spazialmente remote ed esprimendo la loro interdipendenza come deviazione statistica rispetto a gruppi neuronali distribuiti sull'intera corteccia.

Diversi approcci di analisi multivariata sono applicabili allo studio di queste matrici di correlazione, tuttavia uno degli approcci più che si è dimostrato più fruttuoso e quello della teoria delle reti complesse, di cui la teoria dei grafi è il fondamento teorico
La moderna teoria delle reti complesse ha infatti dimostrato che dal punto di vista funzionale, la corteccia cerebrale umana è strutturata sotto forma di una rete dove alcune aree favoriscono e supportano lo scambio di informazione (dette ``hubs''), le quali a loro volta, integrano ed elaborano input provenienti da aree localmente dedite al calcolo di input meno strutturati, tipicamente sensoriali.
Questa rete complessa è stata analizzata con una moltitudine di misure, che hanno mostrato una serie di proprietà notevoli.
Ad esempio è stato osservato che le reti cerebrali, dovendo essere racchiuse in un volume finito, mirano ad ottimizzare l'efficienza di scambio di informazione attraverso la minimizzazione del numero di passi che l'informazione deve attraversare per raggiungere diverse aree. Contemporaneamente, esse mostrano sia a livello strutturale che funzionale una struttura modulare che ne incrementa la robustezza a perturbazioni esterne, quali insulti fisici, patologie o alterazioni durante sviluppo neurale.

Inoltre, la modularità di queste reti funzionali, spiega la capacità del cervello di elaborare una grande quantità di input paralleli provenienti in ogni momento dai vari organi di senso, in quanto ognuno di questi moduli è altamente efficiente nel suo compito e le connessioni fra moduli consentono un trasporto di informazione efficiente.
Diversi sistemi complessi in natura esibiscono proprietà di questo tipo e la teoria delle reti complesse fornisce potenti strumenti per la loro indagine, ispirati alla fisica dei sistemi disordinati, metodi noti come community detection.

Non tutti gli algoritmi di community detection proposti negli anni sono però capaci di districare e rendere evidente la struttura modulare, evidenziando la presenza di sottomoduli a diverse scale.
Uno dei principali problemi che affliggono lo studio delle cosiddette comunità, folti gruppi di agenti fortemente interagenti, ma debolmente aggregati fra diversi gruppi, è il cosiddetto ``limite di risoluzione''.
Questo problema limita la capacità dei metodi community detection di individuare moduli le cui dimensioni siano minori di una scala determinata dalla dimensione generale della rete in esame.

In questo lavoro mi sono occupato esplicitamente di analizzare gli effetti del limite di risoluzione sull'analisi delle reti di connettività funzionale e di fornire una soluzione che permetta una descrizione multiscala dei moduli di queste reti.
A tal fine nel primo e secondo capitolo, dopo una prima parte introduttiva necessaria a guidare il lettore nel lessico della teoria delle reti, ho mirato a descrivere dal punto di vista teorico gli effetti del limite di risoluzione ed ad individuarne gli effetti su reti di connettività reale presenti in letteratura.

Nel terzo capitolo ho investigato gli effetti di tale limite negli studi di neuroimaging, trovando che una varietà di studi dove la teoria delle reti è stata applicata a dataset di connettività funzionale, mostravano i chiari effetti del limite di risoluzione nelle loro analisi.
Per questo motivo ho introdotto il metodo basato sull'ottimizzazione numerica della funzione costo nota come Surprise, centrale per questa tesi, studiandone le proprietà teoriche in dettaglio e trovando che essa non soffre del limite di risoluzione.
I risultati ottenuti dalla massimizzazione di Surprise su dati reali, mostrano che dal punto di vista funzionale le reti di connettività funzionale esibiscono una struttura modulare caratterizzata da moduli di dimensione eterogenea, differentemente da quanto osservato utilizzato approcci a risoluzione limitata.
Alla luce della diversa e più fine partizione in comunità è stato rivalutato il ruolo dei singoli nodi all'interno della rete, facendo emergere l'indicazione che alcune aree precedentemente inosservate (in particolare il precuneuo) svolgono un ruolo fondamentale nell'integrazione e scambio dell'informazione a livello corticale.

In un secondo studio, contenuto nel capitolo quarto, ho applicato i nuovi metodi di analisi sia a reti sintetiche dove abbiamo simulato gli effetti dei fattori di disturbo in fMRI, che a coorti di pazienti affetti da disturbi dello spettro schizofrenico, dove la maggiore risoluzione del mio metodo ha permesso una delle prime osservazioni della quasi completa riorganizzazione funzionale delle aree di input sensoriale, soprattutto nella corteccia temporale che ipotizziamo possa essere di alcune delle psicosi paranoidi (voci, allucinazioni)  riferite da tali pazienti.
In definitiva, il limite di risoluzione sembra avere mascherato in maniera artefattuale la meravigliosa complessità delle reti di connettività funzionale ed il mio studio è uno dei primi che inizia solamente ora a svelare le cause di questa apparente cecità.

