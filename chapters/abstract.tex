Functional magnetic resonance imaging is the most used tool to map functional properties of the human brain connections. Within the mathematical framework of graph theory, the functional connectome can be modeled as a complex network showing peculiar properties.
Among them, one of the most important is the organization in functional modules reflecting an integration-segregation balance of information processing typical of normal brain functioning.
Such modular and hierachical organization is investigated with graph partitioning methods based on the maximization of global fitness functions. The most widely adopted method though dubbed Newman's Modularity, suffers from a resolution limit, as it fails to detect modules that are smaller than a scale determined by the size of the entire network.
Here I explore the effects of this limitation on the study of brain connectivity networks both from a theoretical and a practical perspective.
I demonstrate that the resolution limit prevents detection of important details of the brain modular structure, thus hampering the ability to appreciate differences between networks and to assess the topological roles of nodes.
Hence, I will show that Surprise, a recently proposed fitness function based on probability theory, does not suffer from these limitations.
Surprise maximization in brain co-activation and functional connectivity resting state networks reveals the presence of a rich structure of heterogeneously distributed modules, and differences in networks' partitions that are undetectable by resolution-limited methods.
Moreover, Surprise leads to a more accurate identification of the network's connector hubs, the elements that integrate the brain modules into a cohesive structure.

Moreover we demonstrate