% MESSO NELLA INTRODUZIONE

The first problem that is addressed in this thesis is therefore how to get rid at least partially of the resolution limit. 
After the demonstration and validation on synthetic benchmarks of a method that mitigates the resolution limit, the second part is devoted to the application to real-world networks. Particular emphasis will be given to a study of the functional differences between resting state fMRI functional networks between two populations of healthy controls and schizophrenic patients. 

In the first chapter, after a brief introduction to the basic terminology of graph theory, I will describe the current approaches to community detection and show how the resolution limit is an intrinsic problem of many of them. After expanding the concepts of the resolution limit, I will then present Surprise optimization as a potential solution, accompanying with examples and validations on some benchmark networks. An extension of Surprise to weighted networks, the most common format of networks encountered in studying functional connectivity, will be the last part of chapter.

The second chapter gives an overview of the modular structure of the brain at different levels, from anatomical to functional. I will continue to show how the resolution limit affected many studies of brain modularity in the literature and to what extent the results obtained by standard methods are different than those obtained with Surprise optimization, illustrating point by point the advantages but also the pitfalls.

The third chapter collects the results obtained on real-world brain networks using the methods introduced in this thesis. The most important result is the application of Asymptotical Surprise optimization to the delineation of functional differences in schizophrenic patients. As the main finding, I will show a functional segregation of some of the major networks of the brain and a general reorganization of the modular structure of the functional connectivity.

All the mathematical details regarding the implementation and the theoretical properties of the methods are collected in an appendix, together with a series of possible tuning that may be in order in future studies.

\section*{Selected publications}
Most of the work presented in this thesis has been submitted or published in journals and conferences. My original contribution~\cite{nicolini2016} is the first demonstration that Surprise is a suitable quality function for community detection on brain functional connectivity.
The paper also shows numerically that Surprise is resolution-limit free in the range of network size of interest in brain functional connectivity.
The second published contribution~\cite{nicolini2017} is the application of an extension of community detection based on Surprise, to weighted networks, together with an analysis that indicates that the resolution limit played an important role as it was hiding from view many important functional structures.