\lettrine{S}{t}arting from 19th century’s neuron doctrine, we know that the brain is an incredibly complex system of intercommunicating elements, the neurons, intertwined with other cells that support neuronal activity. 
The dynamical processes running on the biological substrate of neurons are the basis of information processing that gives rise to cognitive and psychological processes, from the perception of the external environment to the representation of mental states and emotions.
Neuroscience tackles the investigation of the brain, the most complex organ in nature, at different scales. The molecular and cellular neuroscience is interested at the finest level of working of the nervous system cells. At this scale, molecular biology and genetics are the tools of investigation, as the interest is mainly addressed to very small populations of neurons. 
Larger number of neurons, though needs a more comprehensive view to identify diseases and pathological states. This view must shift from the microscopic level to the so-called system neuroscience, that deals with highly connected networks of neuron, dubbed neuronal circuits.

The interest of systems neuroscience is how these large-scale circuits form and take shape during the life of the organism, how they produce basic processes like reflexes, motor coordination, circadian rhythms and, ultimately, how they give rise to overt behavioral and different internal neural states like motivation, emotions, and cognition.
The extension of systems neuroscience from the study of small groups of neurons to the whole brain will require radically different computational and experimental tools.

Very small scale neuronal circuits that can be studied by mathematically accurate models are formed by several hundreds neurons. The typical size of the smallest identified functional units, the cortical columns, is around hundred thousand neurons and, at that level, many approximation are needed to approximately reproduce the electrical and chemical activity of neurons, at the point that a comprehensive differential equations model of cortical columns is still lacking. 

Finally, a large scale understanding of the incredibly complicated network of all the neurons in the human brain, the human connectome, remains an idealistic (and hotly debated) target to be reached in the future. This scientific, technological and cultural challenge is named after the term ``connectomics'', and aims to map all the connections in the brain at different scales of resolution. Whether will help answer some of the deepest questions about the human brain, it is not entirely known and still a topic of hot debate.

This effort takes advantage of well-consolidated technologies for looking at the brain at work. These non-invasive methods of analysis of brain functions include neuroimaging techniques like magnetic resonance imaging (MRI), electroencephalography (EEG) and transcranial magnetic stimulation (TMS). Among these modalities, MRI, has emerged as a dominant non-invasive imaging method because it enables the investigation of both brain structure and function with a good compromise in terms of spatial and temporal resolution.

At the morphological level, MRI provides a comprehensive view of the brain, as it can identify the white matter fiber pathways connecting gray matter regions otherwise impossible to obtain in-vivo. The pattern of white matter fibers is often called structural connectivity as it relates to the morphological features of the brain connections. With a metaphor, structural connectivity is the analog to the electrical wirings of the different parts of a microprocessor.

Since its inception, the evolution of technologies in MRI, lead to fMRI, a way to spatially map temporal changes of oxygenated blood concentration in the whole brain. This tool helped scientists to unveil another pattern of connectivity that exists in the brain; less substantial than structural connectivity, but nonetheless important as it reveals the dynamics of the brain at work. The term functional connectivity names the pattern of temporal fluctuations of neuronal activities from different brain regions that give rise to changes in blood oxygenation dynamics.

Considered over the large scale, functional connectivity is descriptive of the temporal metabolic demands of large groups of cortical and subcortical structures, a quantity strictly related to the electrical activity of neurons, and yields precious information about the brain not only on its surface but also in other deeper regions, otherwise difficult to reach.
Throughout the years, fMRI helped neuroscientists and neurologists to enrich the knowledge of the fundamental aspects of the normal brain functional organization. Activations maps have been estimated for almost every kind of cognitive task and together with the lesion studies accumulated over two centuries, they greatly increased the awareness of many aspects of brain organization.

Although conceptually simpler than activations studies, even more interesting is the discovery that when a person is cognitively at rest, the brain engages in a characteristic pattern of dynamic neural activity and most brain regions show spontaneous oscillations in the BOLD signal. These fluctuations show correlations that define a network of spatially remote but similar patterns of signal. These networks consisting in highly correlated subsets of brain regions are altered in neuropsychiatric diseases such as autism and schizophrenia.

One of the most important ideas upon which this work is based is that, is possible to quantify the extent of alteration of the aforementioned resting state networks in the diseased brain using the theoretical framework of complex networks science. Complex network science offers the best tools to assess the topology and organization of brain connectivity networks, as well as powerful means to establish, in a completely data-driven manner, diagnoses of common neurological diseases and additionally design lines of intervention and prevention. Under this light, the brain is seen as a network whose elements are the brain regions at different scales, interconnected by a number of links, describing the strength of interregional coordination.

Strong experimental evidence shows that the nervous system in animals and humans, from the neuronal level up to macroscopic levels, displays an organization that favors the exchange of information over a communication backbone supported by well-connected areas, the hubs of the network, responsible in turn for sending information to local processors. The functional brain network in humans and other animals, consist of connected subnetworks (or modules) associated with cognitive functions, of vital importance for the daily life and the interaction in the environment.

While only the biology of the nervous system can detail the etiology of neurological diseases, their observable symptoms often result in a partial or total reorganization of the functional and structural architecture of the brain. In this respect, the definition and validation of actual data-driven biomarkers based on the effects of the disease on the brain is, ultimately, a highly desirable target to reach within the next years.

Among the network level changes observed in the diseased brain, the alteration of the hierarchical and modular structure of both structural and functional brain connectivity is the most recognizable. Using the tools of graph theory, a branch of combinatorial mathematics describing the properties of structures defined by interconnected entities, the evaluation of functional alterations in the modular structure of FC networks is cast to the problem of community detection in graphs, i.e. the problem of identifying independent groups of brain regions that exhibit strong interconnection between each other but that are weakly connected to other modules. Once the graph is partitioned into smaller groups (or communities) of functionally connected modules, it’s, therefore, simpler to quantify the effect of the disease on the partitioning, as well as the areas involved in such reorganization and the role of the single nodes inside their modules.

A methodological problem that passed largely unnoticed in the neuroscience literature of the last years comes from a fundamental issue in community detection and is known as the resolution limit. This barrier has its roots in the theoretical foundations of community detection and practically implies that the detection of brain modules smaller than a scale determined by the size of the total network in exam is very difficult, or in the worst case, even impossible. This limit hampers the ability to detect small scale structures of functionally correlated brain areas and, last, to notice alterations in connectivity between groups of patients and healthy control. Such limit is therefore of impediment to an implementation of a clinical-level biomarker of brain disease.

This work analyzes the implications of the resolution limit in real-world functional connectivity networks and presents a solution to this issue.
The algorithms and analysis methods developed and detailed in the following chapters give an alternative view on the modular structure of the brain and show that, the human brain functional networks present a rich variety of large and small subnetworks, and that the emergence of a heterogeneous distribution of size of functional modules, takes part in shaping the adaptive behavior of the dynamic brain.

The first problem that is addressed in this thesis is therefore how to get rid at least partially of the resolution limit. 
After the demonstration and validation on synthetic benchmarks of a method that mitigates the resolution limit, the second part is devoted to the application to real-world networks. Particular emphasis will be given to a study of the functional differences between resting state fMRI functional networks between two populations of healthy controls and schizophrenic patients. 

In the first chapter, after a brief introduction to the basic terminology of graph theory, I will describe the current approaches to community detection and show how the resolution limit is an intrinsic problem of many of them. After expanding the concepts of the resolution limit, I will then present Surprise optimization as a potential solution, accompanying with examples and validations on some benchmark networks. An extension of Surprise to weighted networks, the most common format of networks encountered in studying functional connectivity, will be the last part of chapter.

The second chapter gives an overview of the modular structure of the brain at different levels, from anatomical to functional. I will continue to show how the resolution limit affected many studies of brain modularity in the literature and to what extent the results obtained by standard methods are different than those obtained with Surprise optimization, illustrating point by point the advantages but also the pitfalls.

The third chapter collects the results obtained on real-world brain networks using the methods introduced in this thesis. The most important result is the application of Asymptotical Surprise optimization to the delineation of functional differences in schizophrenic patients. As the main finding, I will show a functional segregation of some of the major networks of the brain and a general reorganization of the modular structure of the functional connectivity.

All the mathematical details regarding the implementation and the theoretical properties of the methods are collected in an appendix, together with a series of possible tuning that may be in order in future studies.

\section*{Selected publications}
Most of the work presented in this thesis has been submitted or published in journals and conferences. My original contribution~\cite{nicolini2016} is the first demonstration that Surprise is a suitable quality function for community detection on brain functional connectivity.
The paper also shows numerically that Surprise is resolution-limit free in the range of network size of interest in brain functional connectivity.
The second published contribution~\cite{nicolini2017} is the application of an extension of community detection based on Surprise, to weighted networks, together with an analysis that indicates that the resolution limit played an important role as it was hiding from view many important functional structures.